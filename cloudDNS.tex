\section{DNS-Cloud-Dienste}
Um den Domain Name Service als Anbieter von Ressourcen zu nutzen wird ein DNS-Dienst benötigt. Über diesen Dienst ist es möglich Domain Namen zu registrieren, Zonen zu erstellen und zu verwalten. Für Anbieter von Cloud Dienstleistungen kann die richtige Wahl eines oder mehrerer DNS-Webservice Anbieters ein kritisches Erfolgskriterium sein.

Der offensichtlichste Vorteil eines guten DNS-Dienstleisters ist die Geschwindigkeit. Eine gute DNS-Infrastruktur ermöglicht eine schnelle und zuverlässige Auflösung des Domain-Namens. Je mehr DNS-Server zu Verfügung gestellt werden und umso besser die geographische Abdeckung ist, desto wahrscheinlicher ist es, dass einer der Nameserver in der Nähe des aufrufenden Clients steht. So kann die Reaktionszeit beim Aufruf einer Cloud-Dienstleistung verbessert werden. \cite{Stratusly.2017}

Ein weiteres Kriterium ist das Thema Sicherheit. DNS-Server sind durch die Notwendigkeit eines öffentlichen Zugriffs und ihre Aufgabe als Leitstelle ein beliebtes Ziel für Hacker Angriffe. So könnten durch unbefugten Zugriff auf die Datenbank, Änderungen an den IP-Adressen vorgenommen werden und den Nutzer unwissend auf einen infizierten Server leiten. Auf diese Weise wurden im April 2018 hohe Mengen der Kryptowährung Ether über den Internetdienst MyEatherWallet illegal auf ein fremdes Konto überwiesen \cite{MEWForce.2018}. Dieser Vorfall hat zu enormen Monetären Schäden einzelner Nutzer und Imageverlust des Dienstleisters geführt. \cite{ZDNet.2018}

Angreifer müssen jedoch nicht unbedingt Zugriff auf die Datenbestände erhalten um einen gewaltigen Schaden anzurichten. Ein Distributed-Denial-of-Service (DDoS) Angriff, der eine Überlastung des DNS-Servers anrichtet, würde es unmöglich machen die Adresse der Cloud-Dienstleistung aufzulösen und somit wäre der Service nicht mehr erreichbar. Bei einer kritischen oder viel besuchten Cloud-Anwendung könnte ein solcher Vorfall ebenfalls hohe Umsatzausfälle und Schäden zur Folge haben.

Mit der Hilfe eines erfahrenen DNS-Dienstleisters der auf hohe Sicherheitsstandards wert legt und einer großen und skalierbaren Infrastruktur kann diesen Gefahren entgegengewirkt werden. So erschwert eine Infrastruktur mit vielen geographisch verteilten DNS-Servern, die auf der Anycast-Technologie basiert, die Möglichkeit eines erfolgreichen DDoS Angriffs. Da für einen solchen Angriff ein Netzwerk aus vielen Angreifern benötigt wird und diese sich meist nicht am gleichen geographischen Ort befinden, würde sich die Last auf die jeweils nächsten Server verteilen, da die Möglichkeit einen einzelnen Server gezielt über die IP-Adresse anzusprechen erheblich erschwert wird. \cite{Moura.2016}

Eine Technik die häufig im Zusammenhang mit DNS Verwendung findet ist Redundanz. Die Vermeidung eines Single Point of Failure bewirkt eine Verbesserung der Ausfallsicherheit. Jedoch verschafft sie nicht nur einen Vorteil des möglichen Zugriffs im Fall eines Serverausfalls, sondern verbessert zugleich die Zugriffszeit bei einer strategisch sinnvollen geographischen Verteilung der Server. Aufgrund dessen ist es Sinnvoll, die Nutzung eines zweiten DNS-Providers in Erwägung zu ziehen, um bei Ausfall des primären DNS-Netzwerks ein Failover zum zweiten Anbieter zu ermöglichen.\cite{Stratusly.2017}

Als Anbieter oder Nutzer einer oder mehrerer Cloud-Dienstleistungen ist die Interaktion mit dem DNS-Provider über ein Application-Programming-Interface (API) von großem Nutzen. Insbesondere, bei einer großen Anzahl an Servern, Domains und Zonen, kann die Einrichtung und Pflege dieser einen großen Arbeitsaufwand darstellen. Große Unternehmungen verwalten häufig tausende Domains und eine Schnittstelle für die automatische Verwaltung ist hier von großem Vorteil und birgt die Möglichkeit viele zu Kosten sparen.
