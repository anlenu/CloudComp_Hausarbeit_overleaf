\section{Fazit}
Das Domain-Name-System ist einer der wichtigsten Bestandteile der heutigen Kommunikation. Der Ausfall eines solchen Systems kann schwerwiegende Folgen für Unternehmen haben. Die Verwendung eines Cloud-Dienstes ist hier sehr sinnvoll, da diese meist über eine große globale Infrastruktur verfügen, die für eine schnelle Adressauflösung sorgen kann jedoch für einzelne Unternehmen sehr kostenintensiv im Unterhalt wäre.

Der Aufbau der verschiedenen untersuchten Cloud-Dienste weicht nicht weit voneinander ab und auch die Leistungsunterschiede und preisliche Differenz ist nicht übermäßig relevant. Das erstellte Modell bildet die Kernfunktionalitäten der unterschiedlichen DNS-Cloud-Dienste ab. Es ermöglicht Nutzern eine bessere Übersicht über die Funktionen und den Aufbau eines solchen Services. 

\subsection{Persönliche Errungenschaften}
Während der Erstellung des Modells hatte ich die Möglichkeit mich näher mit dem Thema DNS und Cloud-Dienste zu beschäftigen. Aufgrund meines geringen Vorwissens konnte ich viel über die beiden Themenbereiche lernen und auch erste praktische Erfahrungen im Umgang mit Cloud-Services sammeln. Da ich im Voraus ebenfalls wenig Expertise in der Erstellung von Modellen hatte, konnte ich diese an einem praktischen und angewandten Beispiel ebenfalls vertiefen.
