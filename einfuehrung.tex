\section{Einführung}
Im Umfeld der IT bedarf die Cloud-Techonologie heutzutage kaum noch einer Vorstellung. Sie ist über die letzten Jahre ein alltägliches Werkzeug geworden und auch außerhalb dieses Nutzerkreises gibt es wahrscheinlich kaum noch Menschen, die bisher keinen Berührungspunkt mit einem dieser Services zu verzeichnen haben. Jedoch ist für den normalen Anwender ein Cloud-Dienst nicht immer als solcher zu erkennen oder er weiß aufgrund fehlenden technischen Know-Hows überhaupt nicht, dass er diesen in seinen alltäglichen Tätigkeiten mehrfach verwendet.

Einen solchen, weniger berühmten Service, repräsentiert wahrscheinlich der DNS-Cloud-Service. Trotz des täglichen Gebrauchs großer Netzwerke, wie das des Internets und der Verwendung von URLs wissen die wenigsten, was sich hinter den drei großen Buchstaben verbirgt.

\subsection{Aufgabenstellung}
Die vorliegende Ausarbeitung stell die Abschlussarbeit des Seminars Cloud-Computing im Sommersemester 2018 an der Hochschule für angewandte Wissenschaften Würzburg-Schweinfurt dar, welches im Masterstudiengang Informationssysteme angeboten wurde. Das Ziel der Arbeit ist es  ein (Meta-) Modell zur Beschreibung des DNS-Cloud-Services zu erstellen. Dieses soll den Service aus der Sichtweise eines Nutzers beschreiben, wobei es an mindestens zwei Cloud-Anbietern evaluiert werden soll. Im Fokus stehen hierbei unter anderem die Dienste des Cloud-Service-Provider Amazon AWS.

\subsection{Aufbau der Arbeit}
Die Gliederung ist so gewählt, dass die folgenden Kapitel zunächst allgemeines Wissen zum Thema DNS aufbauen. Daraufhin sollen die Vorteile des Domain-Name-Systems als Cloud-Dienst aufgezeigt und anschließen die Auswahl der betrachteten Cloud-Provider begründet werden. Folgend wird dann das anhand der genannten Cloud-Dienste erarbeitete Modell präsentiert und genauer erläutert. Abschließend wird ein kurzes Fazit zur Arbeit gegeben.
